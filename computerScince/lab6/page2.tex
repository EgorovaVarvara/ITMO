\newpage
\rhead{\textbf{http://kvant.mccme.ru}}
\begin{minipage}[t]{0.44\textwidth}
    верстие, высота уровня воды ы сосу-\linebreak
    де над отверстием равна $H$. Началь-\linebreak
    ная скорость истечения $v=\sqrt{2gH}$.\linebreak
    За малый промежуток времени $\Delta t$ из\linebreak
    сосуда вытекает масса воды $\Delta m=$\linebreak
    $=\rho v_1s \Delta t$ ($\rho$ - плотность жидкости;\linebreak
    $\Delta t$ столь мало, что в тесение этого\linebreak
    времени скорость $v_1$ постоянна). За\linebreak
    это время уровень воды в сосуде\linebreak
    уменьшится на величину $\Delta h=\frac{\Delta m}{S}$\linebreak
    ($S$ - площадь дна сосуда). Поэтому\linebreak
    в следующий момент временаи ($t_2=$\linebreak
    $=\Delta t$) скорость истечения меньше\linebreak
    $v_1$ и равна $v_1 - \Delta v=\sqrt{2g(H-\Delta h)}$.\linebreak
    Возведем последнее равенство в квад-\linebreak
    рат:
    $$v^2_1-2v_1\Delta v+(\Delta v)^2=2gH-2g\Delta h$$.
    Пренебрегая величиной $(\Delta v)^2$ и учи-\linebreak
    тывая, что $v^2_1=2gH$, получаем
    $$v_1\Delta v=g\rho v_1\frac{s}{S}\Delta t, \Delta v=g\rho \frac{s}{S}\Delta t$$
    Следовательно,
    $$\frac{\Delta v}{\Delta t}=g\rho \frac{s}{S}=const$$
    (разумеется, для заданных $s, S$ и $\rho$).\linebreak
    А это означает, что скорость исте-\linebreak
    чения линейно зависит от времени.\linebreak
    Поэтому можно считать, что в тече-\linebreak
    ние всего времени $T$, пока вытека-\linebreak
    ет жидкость, скорость истечения ра-\linebreak
    вна среднему значению (за промежу-\linebreak
    ток времени $T$), то есть равна
    $$v_{ср}=\frac{v_1+0}{2}=\sqrt{\frac{gH}{2}}$$.
    Теперь посмотрим, как влияет на\linebreak
    время вытекания воды площадь от-\linebreak
    верстия. За единицу времени из сосу-\linebreak
    да должна вытекать масса жидко-\linebreak
    сти $Q=\rho sc_{ср}$ ($\rho$ - плотность жид-\linebreak
    кости). При этом мы считает, что ско-\linebreak
    рость в данный момент времени во\linebreak
    всех точках отверстия направлена\linebreak
    строго перпендикулярно к плоскости\linebreak
    сечения отверстия. Однако, если вни-\linebreak
    мательно приглядеться к струе, вы-\linebreak
    текающей из круглого малого отвер-\linebreak
\end{minipage}
\begin{minipage}[b]{0.02\textwidth} 
\center
\textcolor{white}{\hbox{1}} 
\end{minipage}
\begin{minipage}[t]{0.44\textwidth}
    стия в тонкой стенке, то можно заме-\linebreak
    тить, что диаметр ее поперечного се-\linebreak
    чения меньше диаметра отверстия.\linebreak
    Струя как бы сжимается. Сжатие\linebreak
    струи происходит из-за того, что жид-\linebreak
    кость подтекает к отверстию со всех\linebreak
    сторон и ее частицы, движущиеся\linebreak
    к отверстию вдоль стенок сосуда,\linebreak
    пройдя отверстие, продолжают сбли-\linebreak
    жаться с осью струи. Только на неко-\linebreak
    тором расстоянии от кромки отвер-\linebreak
    стия, равном примерно радиус от-\linebreak
    верстия, траектории частиц становят-\linebreak
    ся почти параллельными. Сжатие\linebreak
    струи продолжается и дальше, но\linebreak
    уже несравненно медленнее, незамет-\linebreak
    но для глаза. Сечение, при котором\linebreak
    заканчивается резкое видимое сжа-\linebreak
    тие струи, называется сжатым. Отно-\linebreak
    шение площади сжатого сечения $s_2$\linebreak
    к площади сечения отверстия $s_1$ на-\linebreak
    зывается коэффициентом сжатия $\varepsilon$:
    $$\varepsilon = s_2/s_1$$
    Поскольку расстояние между сече-\linebreak
    ниями $s_1$ и $s_2$ мало (как мы уже гово-\linebreak
    рили, оно равно примерно радиусу\linebreak
    сечения $s_1$), то скорость воды во всех\linebreak
    точках сечения $s_2$ практически равна\linebreak
    $\sqrt{2gh}$ ($h$- высота уровня воды\linebreak
    в сосуде в данный момент времени)\linebreak
    и перпендикулярна к плоскости се-\linebreak
    чения. Следовательно, расход воды\linebreak
    $$Q=s_2v_{ср}=\varepsilon s_1v_{ср}=\varepsilon s_1 \cdot \frac{1}{2}\sqrt{2gH}$$
    \null\quadТеперь совсем просто подсчитать\linebreak
    время, в течение которого вытекает\linebreak
    вода из сосуда. Объем вытекающей\linebreak
    воды $V=SH$, следовательно,
    $$T=\frac{V}{Q}=\frac{SH}{\frac{1}{2}\varepsilon s_1\sqrt{2gH}}=\frac{S\sqrt{2gH}}{\varepsilon s_1\sqrt{g}}$$
    Это соотношение можно проверить\linebreak
    на опыте. Для эксперимента возь-\linebreak
    мите сосуд с отверстием диаметром\linebreak
    $2-3 мм$. Можно воспользоваться ме-\linebreak
    талиической банкой с высокими стен-\linebreak
    ками или цилиндрическим ведром.\linebreak
    Отверстие должно находиться доста-\linebreak
    точно далеко от дна и от поверхности\linebreak
\end{minipage}
\rfoot{\textbf{43}}
\lfoot{\textbf{}}