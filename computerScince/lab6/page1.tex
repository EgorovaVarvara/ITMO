\renewcommand{\footnoterule}{\vspace*{-3pt} %переопределение сносок
    \hrule \columnwidth
    \vspace*{2.6pt}}
\newpage
\rhead{\textbf{http://kvant.mccme.ru}}
\begin{minipage}[t]{0.44\textwidth}
	\resizebox{\textwidth}{0.25\textheight}{\includegraphics*{pic1.png}} %задается размер объекта
        \\
        \\
        Рис. 2.\\
        \\
	ход*) воды, чем цилиндрические (при\linebreak
        одинаковых площадях выходных се-\linebreak
        чений). Это оптимальный угол для\linebreak
        получения наибольшего расхода во-\linebreak
        ды. Конические сходящиеся насадки\linebreak
        дают сплошную струю с большими\linebreak
        скоростями и поэтому широко при-\linebreak
        меняются на практике в брандспой-\linebreak
        тах, соплах гидромониторов.\\
        \null\qquadОбратимся теперь к истечению во-\linebreak
        ды через отверстие в стенке сосуда.\\
        \null\qquadВозьмем заполненный водой ци-\linebreak
        линдрический сосуд, в боковой стен-\linebreak
        ке которого имеется малое отверстие\linebreak
        площадью $s_1$ (рис. 3). Через это от-\linebreak
        верстие вода вытекает под давлением\linebreak
        столба жидкости высоты $H$. Отвер-\linebreak
        стие считается малым, если его раз-\linebreak
        меры по крайней мере в десят раз\linebreak
        меньше высоты столба $H$, создающе-\linebreak
        го напор. При этом условии все точки\linebreak
        малого отверстия находятся прибли-\linebreak
        зительно на одной и той же глубине\linebreak
        от поверхности жидкости, и скорости\linebreak
        течения во всех точках можно счи-\linebreak
        тать одинаковыми.\\
        \null\qquadОтверстием в тонкой стенке назы-\linebreak
        вается отверстие, края которого имеют\linebreak
        достаточно острую кромку, чтобы\linebreak
        толщина стенки не влияла на форму\linebreak
        и условия истечения струи. Это ус-\linebreak
        ловие выполняется в том случае,\linebreak
        когда толщина кромки меньше трех\linebreak
        диаметров отверстия.
        \\
        \\
        \\
        \footnote[0]{*) Расходом называют количество воды,\linebreak
        вытекающей в единицу времени.}%сноска
\end{minipage}
\begin{minipage}[t]{0.02\textwidth} %разделение между колонками текста
\center
\textcolor{white}{\hbox{1}} 
\end{minipage}
\begin{minipage}[t]{0.44\textwidth}
	\resizebox{\textwidth}{0.25\textheight}{\includegraphics*{pic2.png}}
        \\
        \\
        Рис. 3.\\
        \\
	\null\qquadОт чего зависит скорость, с кото-\linebreak
        рой вытекает вода через малое отвер-\linebreak
        стие в тонкой стенке сосуда? Чтобы\linebreak
        ответить на этот вопрос, воспользуем-\linebreak
        ся законом сокранения энергии.
        \null\qquadПусть за малый промежуток вре-\linebreak
        мени $\Delta t$ из сосуда вытекло неболь-\linebreak
        шое количество воды $\Delta m$. Потен-\linebreak
        циальная энергия воды в сосуде при\linebreak
        этом уменьшилась на величину $\Delta mgH$\linebreak
        (время достаточно мало, чтобы счи-\linebreak
        тать уровень $H$ воды в сосуде не ме-\linebreak
        няющимся). Это изменение потенци-\linebreak
        альной энергии равно кинетической\linebreak
        энергии массы воды $\Delta m$, вытекаю-\linebreak
        щей из отверстия со скоростью $v$,\linebreak
        то есть $\frac{\Delta m{v}^2}{2} = \Delta mgH$. Отсюда\linebreak
        $$v = \sqrt{2gH}$$
        \null\qquadСколько времени будет вытекать\linebreak
        вода из сосуда? Очевидно, это зави-\linebreak
        сит от скорости истечения и от пло-\linebreak
        щади отверстия. Разберемся сначала\linebreak
        со скоростью.\\
        \null\qquadКак мы уже показали, $v=\sqrt{2gH}$.\linebreak
        Но эту скорость можно считать по-\linebreak
        стоянной лишь на протяжении ма-\linebreak
        лого промежутка времени (пока мож-\linebreak
        но считать $H=const$). По мере исте-\linebreak
        чения жидкости уровень понижается,\linebreak
        и следовательно, скорость истечения\linebreak
        уменьшается от $v_0=\sqrt{2gH}$ (в на-\linebreak
        чальный момент времени $t=0$) до\linebreak
        $v_T=0$ ($T$ - время вытекания). Каков\linebreak
        характер изменения скорости со вре-\linebreak
        менем? Пусть в начальный момент\linebreak
        времени ($t_1=0$), когда открывают от-\linebreak
\end{minipage}
\lfoot{\textbf{42}}
